\documentclass[letterpaper, 12pt]{article}\usepackage[]{graphicx}\usepackage[]{xcolor}
% maxwidth is the original width if it is less than linewidth
% otherwise use linewidth (to make sure the graphics do not exceed the margin)
\makeatletter
\def\maxwidth{ %
  \ifdim\Gin@nat@width>\linewidth
    \linewidth
  \else
    \Gin@nat@width
  \fi
}
\makeatother

\definecolor{fgcolor}{rgb}{0.345, 0.345, 0.345}
\newcommand{\hlnum}[1]{\textcolor[rgb]{0.686,0.059,0.569}{#1}}%
\newcommand{\hlstr}[1]{\textcolor[rgb]{0.192,0.494,0.8}{#1}}%
\newcommand{\hlcom}[1]{\textcolor[rgb]{0.678,0.584,0.686}{\textit{#1}}}%
\newcommand{\hlopt}[1]{\textcolor[rgb]{0,0,0}{#1}}%
\newcommand{\hlstd}[1]{\textcolor[rgb]{0.345,0.345,0.345}{#1}}%
\newcommand{\hlkwa}[1]{\textcolor[rgb]{0.161,0.373,0.58}{\textbf{#1}}}%
\newcommand{\hlkwb}[1]{\textcolor[rgb]{0.69,0.353,0.396}{#1}}%
\newcommand{\hlkwc}[1]{\textcolor[rgb]{0.333,0.667,0.333}{#1}}%
\newcommand{\hlkwd}[1]{\textcolor[rgb]{0.737,0.353,0.396}{\textbf{#1}}}%
\let\hlipl\hlkwb

\usepackage{framed}
\makeatletter
\newenvironment{kframe}{%
 \def\at@end@of@kframe{}%
 \ifinner\ifhmode%
  \def\at@end@of@kframe{\end{minipage}}%
  \begin{minipage}{\columnwidth}%
 \fi\fi%
 \def\FrameCommand##1{\hskip\@totalleftmargin \hskip-\fboxsep
 \colorbox{shadecolor}{##1}\hskip-\fboxsep
     % There is no \\@totalrightmargin, so:
     \hskip-\linewidth \hskip-\@totalleftmargin \hskip\columnwidth}%
 \MakeFramed {\advance\hsize-\width
   \@totalleftmargin\z@ \linewidth\hsize
   \@setminipage}}%
 {\par\unskip\endMakeFramed%
 \at@end@of@kframe}
\makeatother

\definecolor{shadecolor}{rgb}{.97, .97, .97}
\definecolor{messagecolor}{rgb}{0, 0, 0}
\definecolor{warningcolor}{rgb}{1, 0, 1}
\definecolor{errorcolor}{rgb}{1, 0, 0}
\newenvironment{knitrout}{}{} % an empty environment to be redefined in TeX

\usepackage{alltt}
%\usepackage[nomarkers]{endfloat} %%%%%%%%%
\usepackage{calc}
\usepackage{color}
\usepackage{amsmath,amsthm,amssymb}
\usepackage{graphicx}
\usepackage{float}
% Create new "listing" float
\newfloat{listing}{tbhp}{lst}%[section]
\floatname{listing}{Listing}
\newcommand{\listoflistings}{\listof{listing}{List of Listings}}
\floatstyle{plaintop}
\restylefloat{listing}
 
\usepackage{natbib}
%\usepackage{multind}
\usepackage{booktabs}
\usepackage{enumerate}
\usepackage{todonotes}
% \usepackage{uarial}
% \renewcommand{\familydefault}{\sfdefault}
%%%%%%%%%%%%%%%%%%%%%%%%%%%%%%%%%% to change %%%%%%%%%%%%%%%%%%%%%%%%%%%%%%%%%%%
%% \usepackage{lineno}
%% \linenumber
\renewcommand{\baselinestretch}{1}  %% 2

\usepackage{color,xcolor}
\definecolor{link}{HTML}{004C80}

\usepackage[labelfont=bf]{caption}
\usepackage[english]{babel}
\usepackage[
  pdftex,
  plainpages=false,
  pdfpagelabels,
  pagebackref=true,
  colorlinks=true,
  citecolor=link,
  linkcolor=link
]{hyperref}
\hypersetup{colorlinks,urlcolor=link}
\usepackage{array, lipsum}

% \usepackage{datetime}
\usepackage[margin=2cm,textwidth=19cm]{geometry}
\usepackage{float}



\newcommand{\eg}{{e.\,g.\,}}
\newcommand{\ie}{{i.\,e.\,}}
\newcommand{\pkg}[1]{\textit{#1}}


\newcommand{\N}{\mathcal{N}}

\setlength\parindent{0pt}

%\newcommand{\todo}[1]{\textcolor{red}{#1}}

\makeatletter
\DeclareRobustCommand*\textsubscript[1]{%
  \@textsubscript{\selectfont#1}}
\def\@textsubscript#1{%
  {\m@th\ensuremath{_{\mbox{\fontsize\sf@size\z@#1}}}}}
\makeatother
\IfFileExists{upquote.sty}{\usepackage{upquote}}{}
\begin{document}
\begin{center}
  {\noindent \LARGE \bf Code documentation for the simulation study:\\[2mm]
    Comparison of confidence intervals summarizing\\[2mm]
    the uncertainty of the combined estimate of a meta-analysis
  }\\
\bigskip
{\noindent \Large Felix Hofmann
}\end{center}
\bigskip
\vspace*{.5cm}

\tableofcontents

\newpage 

\section{Overview}

This document gives an overview over the code that implements the
simulation study \emph{Comparison of confidence intervals summarizing the
uncertainty of the combined estimate of a meta-analysis}. The code is split into
six different files:

\begin{itemize}
  \item \texttt{utils.R}
  \item \texttt{study\_simulation.R}
  \item \texttt{studies2cis.R}
  \item \texttt{cis2measures.R}
  \item \texttt{measures2summary.R}
  \item \texttt{simulation.R}
\end{itemize}

The follwing sections will describe the contents of each of these scripts.

\section{utils.R}



The script \emph{utils.R} contains short utility functions that are used
multiple times throughout the simulation.

\begin{itemize}
  \item \texttt{rep2}
  \item \texttt{error\_function}
  \item \texttt{hMeanChiSqMu\_f}
  \item \texttt{hMeanChiSqMu\_chisq}
\end{itemize}

Currently, we use only the functions \texttt{rep2()} and
\texttt{error\_function()}.

\texttt{rep2()} is an extension of the \texttt{rep()} function in the
\emph{base} package. It works exactly as \texttt{rep()} but it can take an input
vector for the \texttt{each} argument. It's use should become clear in the
example below:

\begin{knitrout}
\definecolor{shadecolor}{rgb}{0.969, 0.969, 0.969}\color{fgcolor}\begin{kframe}
\begin{alltt}
\hlkwd{rep2}\hlstd{(}\hlkwc{x} \hlstd{=} \hlkwd{c}\hlstd{(}\hlstr{"a"}\hlstd{,} \hlstr{"b"}\hlstd{,} \hlstr{"c"}\hlstd{),} \hlkwc{each} \hlstd{=} \hlnum{1}\hlopt{:}\hlnum{3}\hlstd{)}
\end{alltt}
\begin{verbatim}
## [1] "a" "b" "b" "c" "c" "c"
\end{verbatim}
\end{kframe}
\end{knitrout}

\texttt{error\_function()} on the other hand is mainly used for logging. Thus,
it is only used in \texttt{tryCatch}-blocks. The intended use is that in case of
errors, the function stores the last object together with the current parameters
to a file on disk (files \emph{error.rds} and \emph{pars.rds}). It also writes
the error message as well as the iteration number to a text file
(\emph{error.txt}).

\section{study\textunderscore simulation.R}



The script \emph{study\_simulation.R} contains all functions that are related to
the simulation of the individual studies. The file contains the following functions.

\begin{itemize}
  \item \texttt{simRE}
  \item \texttt{pAccept}
  \item \texttt{simREbias}
  \item \texttt{sim\_effects}
\end{itemize}

These functions implement the simulation of the individual studies of a
meta-analysis. In the simulation, we only use \texttt{sim\_effects()}, the main
function, which calls the other functions internally.

This function has two arguments, \emph{pars} and \emph{i}
which corresponds to the current iteration.

\todo{describe arguments}

% The argument \emph{pars} 
% , a \emph{data.frame} with exactly
% one row and columns \emph{k}, \emph{sampleSize}, \emph{effect}, \emph{I2},
% \emph{heterogeneity}, \emph{dist}, \emph{large} and \emph{bias},

The mechanism, how this 


\section{studies2cis.R}

\begin{itemize}
  \item \texttt{get\_classic\_obj\_hc()}
  \item \texttt{get\_classic\_obj\_reml()}
  \item \texttt{get\_classic\_obj\_hk()}
  \item \texttt{tau\_prior\_bm()}
  \item \texttt{get\_classic\_obj\_bm()}
  \item \texttt{get\_classic\_obj()}
  \item \texttt{get\_classic\_ci\_reml()}
  \item \texttt{get\_classic\_ci\_hk()}
  \item \texttt{get\_classic\_ci\_hc()}
  \item \texttt{get\_classic\_ci\_bm()}
  \item \texttt{get\_classic\_pi\_reml()}
  \item \texttt{get\_classic\_pi\_hk()}
  \item \texttt{get\_classic\_interval()}
  \item \texttt{get\_classic\_intervals()}
  \item \texttt{get\_p\_value\_functions()}
  \item \texttt{get\_p\_value\_args()}
  \item \texttt{get\_new\_ci\_gamma()}
  \item \texttt{get\_new\_intervals()}
  \item \texttt{get\_tau2()}
  \item \texttt{get\_ints\_reml()}
  \item \texttt{sim2CIs()}
  \item \texttt{calc\_ci()}
\end{itemize}

The main function here is \texttt{calc\_ci()}, which calls all the other
functions internally.

\todo{Describe arguments, explain structure}

\section{cis2measures.R}
\section{measures2summary.R}
\section{simulation.R}

%\newpage
%\bibliographystyle{apalike}
%\bibliography{biblio.bib}


\end{document}
