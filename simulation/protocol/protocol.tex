\documentclass[letterpaper, 9pt]{article}
%\usepackage[nomarkers]{endfloat} %%%%%%%%%
\usepackage{calc}
\usepackage{color}
\usepackage{amsmath,amsthm,amssymb}
\usepackage{graphicx}
\usepackage{float}
% Create new "listing" float
\newfloat{listing}{tbhp}{lst}%[section]
\floatname{listing}{Listing}
\newcommand{\listoflistings}{\listof{listing}{List of Listings}}
\floatstyle{plaintop}
\restylefloat{listing}
 
%\usepackage{natbib}
%\usepackage{multind}
\usepackage{booktabs}
\usepackage{enumerate}
% \usepackage{uarial}
% \renewcommand{\familydefault}{\sfdefault}
%%%%%%%%%%%%%%%%%%%%%%%%%%%%%%%%%% to change %%%%%%%%%%%%%%%%%%%%%%%%%%%%%%%%%
%% \usepackage{lineno}
%% \linenumber
\renewcommand{\baselinestretch}{1}  %% 2

\usepackage{color,xcolor}
\definecolor{link}{HTML}{004C80}

\usepackage[labelfont=bf]{caption}
\usepackage[english]{babel}
\usepackage[pdftex,plainpages=false,pdfpagelabels,pagebackref=true,colorlinks=true, citecolor=link,linkcolor=link]{hyperref}
\hypersetup{colorlinks,urlcolor=link}
\usepackage{array, lipsum}

% \usepackage{datetime}
\usepackage[margin=2cm,textwidth=19cm]{geometry}
\usepackage{float}



\newcommand{\eg}{{e.\,g.\,}}
\newcommand{\ie}{{i.\,e.\,}}
\newcommand{\pkg}[1]{\textit{#1}}


\newcommand{\N}{\mathcal{N}}

\setlength\parindent{0pt}

\newcommand{\todo}[1]{\textcolor{red}{#1}}

\makeatletter
\DeclareRobustCommand*\textsubscript[1]{%
  \@textsubscript{\selectfont#1}}
\def\@textsubscript#1{%
  {\m@th\ensuremath{_{\mbox{\fontsize\sf@size\z@#1}}}}}
\makeatother






\begin{document}
\begin{center}
  {\noindent \LARGE \bf Simulation protocol:\\[2mm]
    Comparison of confidence intervals summarizing the\\[2mm]
    uncertainty of the combined estimate of a meta-analysis
  }\\
\bigskip
{\noindent \Large Florian Gerber, Leonhard Held, Lisa Hofer, Philip Heesen
}\end{center}
\bigskip
\vspace*{.5cm}


For the present protocol is inspired by \cite{burt:etal:06} and \cite{morr:etal:19}.


The simulation is implemented in \texttt{simulate\_all.R}.


\tableofcontents

\newpage 

\section{Aims and objectives}\label{ref:aims}
Comparison of confidence intervals summarizing the uncertainty of the combined estimate of a meta-analysis.

\begin{enumerate}
\item Considered methods are
\begin{enumerate}
\item DerSimonian and Laird (DL) \cite{ders:lair:86}
\item Hartung Knapp Sidik Jonkman (HK) \cite{IntHoutIoannidis}
\item Harmonic mean analysis with alternative {\texttt none} \cite{Held2020b}
\item Harmonic mean analysis with alternative {\texttt two.sided} \cite{Held2020b}
\item Harmonic mean analysis with alternative {\texttt none}, additive variance adjustment with $\hat I^2$. An extension of the idea in \cite{Held2020b}. 
\item Harmonic mean analysis with alternative {\texttt none}, multiplicative variance adjustment \cite{mawd:etal:17}.
  An extension of the idea in \cite{Held2020b}. 
\item Henmi and Copas (HC) \cite{henm:copa:10}.
\end{enumerate}
\item We assess the CIs using the following criteria
  \begin{enumerate}
  \item CI coverage of combined effect, \ie, the proportion of intervals containing the true effect
  \item CI coverage of study effects, \ie, the proportion of intervals containing the true study specific effect
  \item CI width 
  \item CI score \cite{Gnei:Raft:07}
  \item number of CIs (only for Harmonic mean method with alternative {\texttt none}).
  \end{enumerate}
\end{enumerate}






\section{Simulation procedures}


\subsection{Allowance for failures}
We expect no failures, \ie, for all simulated datasets all type of CI methods should lead to a valid CI and all valid CIs should lead to valid CI criteria.
If a failure occurs, we stop the simulation and investigate the reason for the failure. 




\subsection{Software to perform simulations}
The simulation study is performed using the statistical software R \cite{R}.
We save the output of \texttt{sessionInfo()} giving information on the used version of R, packages, and platform with the simulation results.



\subsection{Random number generator}
We use the package 'doRNG' with its default random number generator to ensure that random numbers generated inside parallel for loops are independent and reproducible.


\section{Scenarios to be investigated} \label{sec:scenario}
The $360$ simulated scenarios consist of all combinations of the following parameters:
\begin{itemize}
\item Higgin's $I^2$ heterogeneity measure $\in \{0, 0.3, 0.6, 0.9\}$.
\item Number of studies summarized by the meta-analysis $k \in \{2, 3, 5, 10, 20\}$.
\item Publication bias is  $\in \{\text{'none'}, \text{'moderate'}, \text{strong'}\}$ following the terminology of \cite{henm:copa:10}.
  The average study effect also influences the publication bias, and we set it to $\theta = 0.2$ to obtain a similar scenario as used in \cite{henm:copa:10}.
\item The distribution to draw the true study values $\delta_i$ is either 'Gaussian' or 't'. The latter leads to more 'outliers'.
\item Increase the number of patients for 0, 1, or 2 studies by a factor 10. 
\end{itemize}
Sample size of the individual studies (number of patients per study) is fixed to $n = 50$. Note that \cite{IntHoutIoannidis} use a similar setup.


\subsection{Simulation of the data}
For each scenario in Section~\ref{sec:scenario} we
\begin{enumerate}
\item simulate $10000$ meta-analysis datasets
\item compute the CI's listed in Section~\ref{ref:aims} for each meta-analysis
\item summarize the performance of the CI's by the criteria listed in Section~\ref{ref:aims}
\end{enumerate}


For the \textbf{Gaussian model without publication bias}, the simulation of one meta-analysis dataset is performed as follows:
\begin{enumerate}
\item Compute the within-study variance $\epsilon^2 = \frac{2}{n}$.
\item Compute the between-study variance $\tau^2 = \epsilon^2 \frac{I^2}{1-I^2}$. 
\item For a trial $i$ of the meta-analysis with $k$ trials, $i = 1, \dots, k$:
  \begin{enumerate}
  \item Simulate the true effect size using the Gaussian model: $\delta_i \sim \N(\theta, \tau^2)$ or using a students-t distribution such that the samples have mean $\theta$ and variance $\tau^2$.
  \item Simulate the effect estimates of each trial $y_i \sim \N(\delta_i, \frac{2}{n})$.
  \item $\text{se}_i \sim \sqrt{\frac{\chi^2(2n-2)}{(n-1)n}}$ are the standard errors of trial outcomes
  \end{enumerate}
\end{enumerate}

\paragraph{Note: Student's t instead of Gaussian}\mbox{}\\
To generate more 'outlier' studies, we use the Student's~t instead of the Gaussian distribution to simulate the true study effects~$\delta_i$. 
We use the Student's~t with 4 degrees of freedom and choose the other parameter such that the variance of the samples is~$\tau^2$. 


\paragraph{Note: Publication bias}\mbox{}\\
To simulate studies under \textbf{publication bias}, we follow the suggestion of \cite{henm:copa:10} and accept each simulated study with probability
$$\exp(-4\, \Phi(-\theta_i / \text{se}_i)^\gamma ),$$
where $\gamma = 3$ and $\gamma = 1.5$ correspond to \emph{moderate} and \emph{strong} publication bias, respectively.
This is, accepted studies are kept and for a rejected study we replace $\theta_i$ and $\text{se}_i$ by newly simulated values, which are then again accepted with the given probability above.
This procedure is repeated until the required number of studies is simulated. 

The mean study effect $\theta$ and the sample size $n$ have an influence on the acceptance probability.
To obtain a similar scenario as in \cite{henm:copa:10} we set
$$ \theta / \sqrt{2/n}  \overset{!}{=} 1 \Rightarrow \theta = \sqrt{2/n}$$
This is, for $n=50$ we use $\theta = 0.2$. See the R function \texttt{simREbias()}.


\paragraph{Note: Unbalanced sample sizes}\mbox{}\\
To study the effect of unbalanced sample sizes we consider the following setup:
\begin{enumerate}
\item Increase the sample size of \textbf{one} of the $k$ by a factor 10. 
\item Increase the sample size of \textbf{two} of the $k$ by a factor 10. 
\end{enumerate}
A possible publication bias is only applied to the small studies.
See the argument \texttt{large} of \texttt{simREbias()}.



\section{Statistical methods to be evaluated} \label{sec:method}
\begin{enumerate}
\item DerSimonian and Laird (DL) \cite{ders:lair:86}
\item Hartung Knapp Sidik Jonkman (HK) \cite{IntHoutIoannidis}
\item Harmonic mean analysis with alternative {\texttt none} \cite{Held2020b}
\item Harmonic mean analysis with alternative {\texttt two.sided} \cite{Held2020b}
\item Harmonic mean analysis with alternative {\texttt none}, additive variance adjustment with $\hat I^2$. An extension of the idea in \cite{Held2020b}. 
\item Harmonic mean analysis with alternative {\texttt none}, multiplicative variance adjustment \cite{mawd:etal:17}. An extension of the idea in \cite{Held2020b}.
\item Henmi and Copas (HC) \cite{henm:copa:10}.
\end{enumerate}


\paragraph{Extension: Multiplicative variance adjustment}\mbox{}\\
For the harmonic mean analysis with alternative {\texttt none}, we also try a variant with multiplicative instead of additive variance adjustment \cite{mawd:etal:17}.
(Note: We think in \cite{mawd:etal:17} $v_i$ and $v_i^2$ are confused.)

\section{Estimates to be stored for each simulation and summary measures to be calculated over all simulations}
For each simulated meta-analysis we store the assessments (Section~\ref{sec:method}) of all the CI methods (Section~\ref{sec:method}).
Then we compute the average performance by taking the mean.


\section{Number of simulations to be performed}
We perform the simulation of each scenario 10000 times.
We repeat the final simulation several times to assess the variability in the simulation results.


\section{Criteria to evaluate the performance of statistical methods for different scenarios}
Assess the CIs by the following criteria 
\begin{enumerate}
  \item CI coverage of combined effect, \ie, the proportion of intervals containing the true effect
  \item CI coverage of study effects, \ie, the proportion of intervals containing the true study specific effect
  \item CI width 
  \item CI score \cite{Gnei:Raft:07}
  \item number of CIs (only for Harmonic mean method with alternative {\texttt none}).
\end{enumerate}



\section{Presentation of the simulation results}
We present the average of each performance measurement in figures. The figures have
\begin{itemize}
\item the number of studies $k$ on the x axis
\item the performance measure on the y axis
\item one connecting line and color for each value of $i^2$
\item one panel for each CI method
\end{itemize}


\newpage
\bibliographystyle{apalike}
\bibliography{biblio.bib}


\end{document}

 
